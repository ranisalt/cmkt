\documentclass{article}
\usepackage[brazil]{babel}
\usepackage[inline]{enumitem}
\usepackage[a4paper,margin=0.5in]{geometry}
\usepackage[utf8]{inputenc}
\usepackage[svgnames]{xcolor}

\begin{document}

\title{Mercado de Capitais I - Lista 1}
\author{Ranieri S. Althoff -- 13100773}
\date{\today}

\maketitle

\begin{enumerate}
    \item Na questão abaixo assinale a alternativa correta.
        Os agentes superavitários:

        \begin{enumerate*}[label={(\Roman*)}]
            \item Possuem poupança
            \item Gastam mais do que sua renda
            \item Podem emprestar para os agentes deficitários
            \item Necessitam de financiamento para cobrir suas despesas.
        \end{enumerate*}

        \begin{enumerate}[label={(\alph*)},leftmargin=1cm]
            \item Somente I está correta
            \item Estão corretas I e II
            \item \textcolor{DarkGreen}{Estão corretas I e III}
            \item Estão corretas II e IV
            \item nda
        \end{enumerate}

    \item Uma das vantagens da intermediação financeira é a diversificação do
        risco. Diga o que você entende por este conceito, exemplificando-o.

        \textbf{Resposta:} a diversificação do risco consiste em comprar uma
        variedade de ativos pelo intermediário, diluindo assim os riscos do
        investimento desde que não haja correlação que implique em uma perda em
        múltiplos títulos. Como exemplo, um investidor com apenas ações da BRF
        perdeu uma quantia considerável com a crise no setor, enquanto um
        investidor com ativos variados teria uma perda amortizada.

    \item Com base no exemplo da página 6 e 7 da apostila, faça os cálculos de
        custos com e sem mercado central para uma “economia simplificada” com
        1.000 (mil) produtos.

    \item As instituições financeiras podem ser divididas em monetárias e não
        monetárias. As primeiras (Banco Central e bancos comerciais) são
        capazes de criar moeda. Exemplifique, através do multiplicador da
        oferta de moeda, como os bancos comerciais podem criar moeda via
        depósito a vista.

    \item O que você entende por mercado primário e secundário? Dê um exemplo
        de dois mercados secundários que você conhece.

        \textbf{Resposta:} o mercado primário é o lançamento de valores
        mobiliários por uma empresa no mercado, e serve para captação de
        recursos pela empresa. O mercado secundário é a negociação dos valores
        mobiliários exclusivamente entre os investidores, onde ocorre a troca
        de propriedade dos títulos. Exemplos de mercados secundários: bolsas de
        valores e mercado de balcão.

    \item Explique as funções do Conselho Monetário Nacional, Banco Central do
        Brasil e da Comissão de Valores Mobiliários.

    \item O que você entende por operações de mercado aberto? Qual a finalidade
        e quais os principais títulos nele negociados? Onde estes títulos são
        registrados?

    \item O que você entende por Mercado de Capitais? Qual a finalidade e os
        principais títulos nele negociados?

    \item Defina e caracterize uma sociedade anônima aberta e fechada.

    \item Quais as principais vantagens de uma sociedade anônima sobre uma
        sociedade limitada?
        \textbf{Resposta:} em uma sociedade anônima, a responsabilidade dos
        acionistas é limitada pelo valor que possuem em ações da empresa, não
        respondendo pelas obrigações assumidas pela sociedade, e é possível
        ceder livremente as ações, não alterando a estrutura da sociedade com a
        entrada ou saída de um novo acionista.

    \item Diferencie uma Sociedade Corretora de Títulos e Valores Mobiliários
        de uma Sociedade Distribuidora.

    \item A Lei 10.303, de 2001, conhecida como Nova Lei das S.A., trouxe
        grandes aprimoramentos na Lei das S.A., de 1976. Cite pelo menos três
        alterações?

    \item Verifique se as alternativas são verdadeiras ou falsas.

        No Brasil, as ações podem ser:

        \begin{itemize}
            \item[(F)] Nominativas
            \item[(V)] Ordinárias
            \item[(F)] Ao portador
            \item[(V)] Preferenciais
            \item[(F)] de Fruição
        \end{itemize}

        Subscrição refere-se a:

        \begin{itemize}
            \item[(V)] Mercado primário
            \item[(F)] Pagamento de dívidas
            \item[(V)] Aumento do capital social
            \item[(V)] Obtenção de recursos pela empresa
        \end{itemize}

        A CVM:

        \begin{itemize}
            \item[(F)] Atua no mercado de crédito
            \item[(F)] É órgão executor da política monetária
            \item[(V)] As bolsas de valores são subordinadas à CVM
            \item[(V)] O mercado de balcão é subordinado à CVM
        \end{itemize}

        Os bancos comerciais:

        \begin{itemize}
            \item[(F)] Podem se organizar na forma de empresa limitada
            \item[(V)] Sâo capazes de criar moeda
            \item[(V)] Captam recursos através dos depósitos à vista
            \item[(V)] Captam recursos através de CDB/RDB
        \end{itemize}

        Representam capital próprio:

        \begin{itemize}
            \item[(F)] Estoque de produtos vendidos
            \item[(V)] Ações preferenciais
            \item[(F)] \textit{Commercial papers}
            \item[(F)] Debêntures
            \item[(V)] Ações ordinárias
            \item[(F)] Duplicatas a receber
        \end{itemize}

        São instituições não-monetárias:

        \begin{itemize}
            \item[(F)] Caixas Econômicas
            \item[(V)] Cooperativas de crédito
            \item[(V)] Corretoras
            \item[(V)] Bancos de investimento
            \item[(F)] Financeiras
            \item[(V)] Empresas de leasing
            \item[(F)] Bancos múltiplos
        \end{itemize}
\end{enumerate}
\end{document}
