\documentclass{article}
\usepackage[brazil]{babel}
\usepackage[inline]{enumitem}
\usepackage[utf8]{inputenc}
\usepackage[svgnames]{xcolor}

\begin{document}

\title{Mercado de Capitais I - Lista 1}
\author{Ranieri S. Althoff -- 13100773}
\date{\today}

\maketitle

\begin{enumerate}
    \item Na questão abaixo assinale a alternativa correta.
        Os agentes superavitários:

        \begin{enumerate*}[label={(\Roman*)}]
            \item Possuem poupança
            \item Gastam mais do que sua renda
            \item Podem emprestar para os agentes deficitários
            \item Necessitam de financiamento para cobrir suas despesas.
        \end{enumerate*}

        \begin{enumerate}[label={(\alph*)},leftmargin=1cm]
            \item Somente I está correta
            \item Estão corretas I e II
            \item \textcolor{DarkGreen}{Estão corretas I e III}
            \item Estão corretas II e IV
            \item nda
        \end{enumerate}

    \item Uma das vantagens da intermediação financeira é a diversificação do
        risco. Diga o que você entende por este conceito, exemplificando-o.

        R: a diversificação do risco consiste em comprar uma quantidade de
        variados títulos primários pelo intermediário, diluindo assim os riscos
        do investimento desde que não haja correlação que implique em uma perda
        em múltiplos títulos. Como exemplo, um investidor com apenas ações da
        BRF perdeu uma quantia considerável com a crise no setor, enquanto um
        investidor com ativos variados teria uma perda amortizada.

    \item Com base no exemplo da página 6 e 7 da apostila, faça os cálculos de
        custos com e sem mercado central para uma “economia simplificada” com
        1.000 (mil) produtos.

    \item As instituições financeiras podem ser divididas em monetárias e não
        monetárias. As primeiras (Banco Central e bancos comerciais) são
        capazes de criar moeda. Exemplifique, através do multiplicador da
        oferta de moeda, como os bancos comerciais podem criar moeda via
        depósito a vista.

    \item O que você entende por mercado primário e secundário? Dê um exemplo
        de dois mercados secundários que você conhece.

    \item Explique as funções do Conselho Monetário Nacional, Banco Central do
        Brasil e da Comissão de Valores Mobiliários.

    \item O que você entende por operações de mercado aberto? Qual a finalidade
        e quais os principais títulos nele negociados? Onde estes títulos são
        registrados?

    \item O que você entende por Mercado de Capitais? Qual a finalidade e os
        principais títulos nele negociados?

    \item Defina e caracterize uma sociedade anônima aberta e fechada.

    \item Quais as principais vantagens de uma sociedade anônima sobre uma
        sociedade limitada?

    \item Diferencie uma Sociedade Corretora de Títulos e Valores Mobiliários
        de uma Sociedade Distribuidora.

    \item A Lei 10.303, de 2001, conhecida como Nova Lei das S.A., trouxe
        grandes aprimoramentos na Lei das S.A., de 1976. Cite pelo menos três
        alterações?

    \item 13. Verifique se as alternativas são verdadeiras ou falsas.
        No Brasil, as ações podem ser:

        \begin{itemize}
            \item[(F)] Nominativas
            \item[(V)] Ordinárias
            \item[(F)] Ao portador
            \item[(V)] Preferenciais
            \item[(F)] de Fruição
        \end{itemize}
\end{enumerate}
\end{document}
